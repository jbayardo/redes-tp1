\documentclass{article}

\usepackage[utf8]{inputenc}
\usepackage[spanish]{babel}

\usepackage{caratula}

\usepackage{subcaption}
\usepackage{graphicx}
\usepackage{subfig}
\usepackage{dirtytalk}
\usepackage{enumerate}

\usepackage{amssymb}
\usepackage{mathtools}
\usepackage{amsmath}
\usepackage{amsthm}

\usepackage{algorithm}
\usepackage{algpseudocode}
\usepackage{listingsutf8}

\usepackage{float}
\floatplacement{figure}{h!}

\usepackage{geometry}
\usepackage{fixltx2e}
\usepackage{wrapfig}
\usepackage{cite}
\usepackage{dsfont}

\usepackage[space]{grffile}

\geometry{
 a4paper,
 total={210mm,297mm},
 left=30mm,
 right=30mm,
 top=30mm,
 bottom=30mm,
 }
 
\newtheorem{theorem}{Teorema}[section]
\newtheorem{corollary}{Corolario}[theorem]
\newtheorem{lemma}{Lema}[theorem]
 
\theoremstyle{definition}
\newtheorem{definition}{Definición}[section]
 
\theoremstyle{remark}
\newtheorem*{remark}{Observación}
 
\begin{document}
% Estos comandos deben ir antes del \maketitle
\materia{} % obligatorio

\titulo{Trabajo Práctico 1}
\subtitulo{}
\grupo{}

\integrante{Bayardo Julián}{850/13}{julian@bayardo.com.ar} % obligatorio
\integrante{Cuneo Christian}{755/13}{chriscuneo93@gmail.com} % obligatorio 
\integrante{Fosco Martín Esteban}{449/13}{chriscuneo93@gmail.com} % obligatorio 
\integrante{Germán Pinzón}{475/13}{pinzon.german.94@gmail.com}
 
\maketitle

\pagebreak

\tableofcontents

\pagebreak

\section{Introducción}

El Trabajo Práctico presentado se analiza y distinguen diversos aspectos de la red, utilizando un enfoque desde la Teoría de la Información.

En particular, se presentarán cuatro experimentos realizados sobre redes distintas, observando primero la incidencia de los distintos protocolos que encapsulan los paquetes de cada red y analizando estos datos.

Luego, nos enfocaremos en los paquetes que usen el protocolo ARP, y procederemos a establecer algún criterio para distinguir algunos nodos por sobre el resto.

\newpage

\section{Criterio para Filtrado de Resultados}

Para la presentación de los datos se decidió obviar los que sean de proba menor a 7e(-4)

\begin{LARGE}

ESTO NO VA ACAAA AJGFDNFKJASFA BUSQUEN LUGAR QUE NO QUEDE MAL
Y ESCRIBIRLO BIEN
\end{LARGE}

\newpage

\section{Primer Experimento: }

\subsection{Condiciones}

\subsection{Resultados}

\subsection{Conclusión}

\newpage

\section{Segundo Experimento: Laboratorios-DC}

\subsection{Método}
En el siguiente experimento, se analizó la red wi-fi del pabellon 1 de Ciudad Universitaria, Laboratorios-DC 

\subsection{Resultados}

\subsection{Conclusión}

\newpage

\section{Tercer Experimento: }

\subsection{Condiciones}

\subsection{Resultados}

\subsection{Conclusión}

\newpage

\section{Cuarto Experimento: }

\subsection{Condiciones}

\subsection{Resultados}

\subsection{Conclusión}

\end{document}